\documentclass[10pt,a4paper]{report}
\usepackage[utf8]{inputenc}
\usepackage{amsmath}
\usepackage{amsfonts}
\usepackage{amssymb}
\usepackage{listings}
\usepackage{enumerate}
\usepackage[T1]{fontenc}
\sffamily
\begin{document}
Pasos a seguir al 8/3/19:

\begin{enumerate}


\item Ejecutar 

\begin{lstlisting}[language=bash]
python main.py
\end{lstlisting}

en la carpeta correspondiente.

\item Ingresar las reallciones (en el readme va a estar el ejemplo que anda por el momento)

\item Por ahora, el usuario tiene que confrimarle al programa que la red es MESSI, e ingresar $S^{(0)}, S^{(1)}$, etc y finalizar con un END CORES

\item El programa chequea la condición $C'$. Por hoy, nos gustarís que pueda chequear si hay un único simple path que una dos vértices de $G_2$. Esperamos, en esta primera versión del programa, trabajar bajo las hipótesis del Teorema 35.
  Por hoy, el usuario tiene que chequear a mano si es s-toric y dar ENTER. Como todavía no podemos encontrar los binomios, la B también se debe ingresar manualmente. (La M también, pero en breve haremos que el programa la calcule sola a partir del grafo.)

\item Chequea si es mixed

\item Aplica el algoritmo del paper de los MESSI systems y devuelve: los ortantes (con vectores testigo en cada ortante) y los testigos $x_1, x_2$, y $\kappa$. En breve mostrará testigos por cada ortante. (Sería ideal que mostrara kappas positivos, pero estamos esperando la respuesta de Carsten.)

\end{enumerate}
\end{document}