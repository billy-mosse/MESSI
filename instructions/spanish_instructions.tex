\documentclass[10pt,a4paper]{report}
\usepackage[utf8]{inputenc}
\usepackage{amsmath}
\usepackage{amsfonts}
\usepackage{amssymb}
\usepackage{listings}
\usepackage{url}
\usepackage{listingsutf8}
\usepackage[spanish]{babel}
\usepackage{dirtytalk}

\begin{document}

Hay varias cosas para hacer. Están itemizadas, con una explicación después de cada tarea:

\lstset{language=bash,
basicstyle=\ttfamily,
extendedchars=true,
inputencoding=latin1,
literate=
{á}{{\'a}}1
{à}{{\`a}}1
{ã}{{\~a}}1
{é}{{\'e}}1
{ê}{{\^e}}1
{í}{{\'i}}1
{ó}{{\'o}}1
{õ}{{\~o}}1
{ú}{{\'u}}1
{ü}{{\"u}}1
{ç}{{\c{c}}}1
}

\begin{itemize}

	\item Instalar git. 
\begin{lstlisting}
#Para tener las versiones de paquetes más recientes
sudo apt-get update
#GIT
sudo apt-get install git
\end{lstlisting}

Dicho mal y pronto, git es un programa usado para trabajar en equipo sobre un proyecto (como overleaf/sharelatex, dropbox, etc), pero con algunas diferencias. La más importante es que si uno hace cambios en los archivos, éstos \textbf{no se actualizan instantáneamente} en la computadora de la otra persona.

Además, podemos trabajar sobre el mismo archivo (en lugares distintos), y luego combinar los cambios de manera muy fácil.

\item Crease una cuenta en \url{https://github.com/}

El proyecto que hice para el final está alojado (hosteado) en un sitio web - el que más se usa - llamado \textbf{github}. Es gratis. Tendrías que hacerte una cuenta ahí así te puedo agregar al proyecto.

(¿Estás de acuerdo con que sea público?)

\item Compartirme tu usuario de github por email, así te agrego al proyecto y podés subir cambios.

\item Elegir carpeta donde trabajar.

En tercer lugar, tenés que elegir una carpeta en tu máquina donde vas tus proyectos de github. Yo uso \textbf{/home/billy/Projects/git/github}, pero no hace falta que sea ni remotamente parecida.

\item Bajarse el proyecto de github.

\begin{lstlisting}
#Estate parada en la carpeta donde vas a trabajar
#La carpeta no debería ser "MESSI"
#porque este comando te baja todo a una carpeta "MESSI"
git clone https://github.com/billy-mosse/MESSI.git
\end{lstlisting}

Deberías ver una carpeta \say{MESSI} con subcarpetas \say{code, garbage, instructions}, etc.

\item Instalar la última versión de \textbf{python3}:

\begin{lstlisting}
#Creo que viene por default ya, igual.
sudo apt-get install python3
#Si ya lo tenías instalado, actualizalo
sudo apt-get upgrade python3
\end{lstlisting}

\item Instalar \textbf{pip}, para bajarse librerías de python, y otros yuyos:

\begin{lstlisting}
sudo apt-get -y install python3-pip
pip3 install --upgrade pip

#Yo tuve un problema y tuve que hacer esto
pip3 install --upgrade setuptools pip
\end{lstlisting}

\item (Opcional, pero recomendado) Instalar \textbf{virtualenv}:

\begin{lstlisting}
#Instalar virtualenv por medio de pip.
python3 -m pip install --user virtualenv
\end{lstlisting}

Virtualenv sirve para cuando tenés más de un proyecto en python. Tal vez para uno necesitás la versión 3.7 de la librería matplotlib, pero otro es medio antiguo y sólo anda con la versión 2.2. Virtualenv hace que puedas no instalar paquetes \textbf{globalmente}.

Cuando uno trabaje en este proyecto, \say{prende} el entorno virtual de python correspondiente, que va a tener instaladas las versiones específicas de las librerías necesarias. (Es un comando muy simple.)

De nuevo, no hace falta, a mí no me cambia nada si lo usás o no. Es más para uno.

\item (Opcional, cont) Crear el virtualenv correspondiente al proyecto, en la misma carpeta del proyecto, o en una carpeta virtualenvs/ en tu home:

\begin{lstlisting}
python3 -m virtualenv env_messi
\end{lstlisting}

¡Avisame si en vez de env\_messi usás otro nombre para el virtualenv, porque en ese caso tengo que hacer algo!

\item Prender el entorno virtual:

\begin{lstlisting}
source env_messi/bin/activate
\end{lstlisting}

\item Instalar las librerías que necesita el proyecto:

\begin{lstlisting}
#El doble AND es para correr comandos en serie (uno detras de otro)
#siempre y cuando mp devuelvan error
pip3 install numpy &&\
pip3 install matplotlib &&\
pip3 install pandas &&\
pip3 install networkx
\end{lstlisting}


\item Instalar sage para python:

Bajarse sage de \url{http://linorg.usp.br/sage/linux/index.html}

Descomprimirlo, y correr:

\begin{lstlisting}
make configure &&\
./configure --with-python=3 &&\
make build
\end{lstlisting}

\end{itemize}



https://packaging.python.org/guides/installing-using-pip-and-virtualenv/
Instrucciones para instalarse el entorno de trabajo:


\begin{lstlisting}
sudo apt-get install python3
\end{lstlisting}

\end{document}